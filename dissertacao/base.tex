\documentclass[12pt,a4paper,oneside]{report}

% -------------------------
% Pacotes essenciais
% -------------------------
\usepackage[utf8]{inputenc}
\usepackage[T1]{fontenc}
\usepackage[english,portuguese]{babel}
\usepackage{lmodern}
\usepackage{geometry}
\usepackage{setspace}
\usepackage{graphicx}
\usepackage{hyperref}

% (Opcional, mas comum em dissertações)
\usepackage{amsmath,amssymb,amsthm}

% -------------------------
% Margens (ajusta se o template UAb pedir outras)
% -------------------------
\geometry{
  left=3cm,
  right=2cm,
  top=3cm,
  bottom=2.5cm
}

\onehalfspacing
\hypersetup{
  colorlinks=true,
  linkcolor=black,
  citecolor=black,
  urlcolor=black
}

% =========================================================
% Metadados (preenche depois)
% =========================================================
\newcommand{\UAbUniversity}{UNIVERSIDADE ABERTA}
\newcommand{\DocTitle}{Formas Normais: da Lógica Matemática à Computação}
\newcommand{\AuthorName}{Olivia Pachele Mattiazzo}
\newcommand{\DegreeName}{Mestrado em Estatística, Matemática e Computação}
\newcommand{\OrientadorUm}{Prof.\textsuperscript{ª} Dr.\textsuperscript{ª} Gilda Ferreira}
\newcommand{\OrientadorDois}{Prof.\textsuperscript{ª} Dr.\textsuperscript{ª} Yves Robert}
\newcommand{\DocMonthYear}{Novembro 2026}

% =========================================================
\begin{document}

% =========================================================
% CAPA 1 (com orientação / coorientação)
% =========================================================
\begin{titlepage}
  \selectlanguage{portuguese}
  \thispagestyle{empty}
  \begin{center}
    {\Large \UAbUniversity \par}
    \vspace{2.5cm}
    {\Large \textbf{\DocTitle}\par}
    \vspace{0.5cm}
    {\Large \AuthorName\par}
    \vspace{0.8cm}
    {\Large \DegreeName\par}
    \vspace{1.2cm}
    {\large Orientadores: \OrientadorUm\par}
    {\large e \OrientadorDois\par}
    \vfill
    {\Large \DocMonthYear\par}
  \end{center}
\end{titlepage}

% =========================================================
% CAPA 2 (sem orientação/coorientação)
% =========================================================
\begin{titlepage}
  \selectlanguage{portuguese}
  \thispagestyle{empty}
  \begin{center}
    {\Large \UAbUniversity \par}
    \vspace{2.5cm}
    {\Large \textbf{\DocTitle}\par}
    \vspace{0.5cm}
    {\Large \DocSubtitle\par}
    \vspace{2.5cm}
    {\Large \AuthorName\par}
    \vspace{0.8cm}
    {\Large \DegreeName\par}
    \vfill
    {\Large \DocMonthYear\par}
  \end{center}
\end{titlepage}

% =========================================================
% FRONTMATTER (numeração romana, como no PDF)
% =========================================================
\pagenumbering{roman}
\setcounter{page}{2} 

% =========================================================
% Creative Commons License (página ii no PDF)
% =========================================================
\selectlanguage{portuguese}
\chapter*{Creative Commons License}
% (texto da licença aqui depois)

% =========================================================
% Agradecimentos (página iii)
% =========================================================
\chapter*{Agradecimentos}
% (texto aqui depois)

% =========================================================
% Página em branco / separador (se precisares)
% \clearpage\null\thispagestyle{empty}\clearpage

% =========================================================
% Dedicatória (no PDF: página com dedicatória curta)
% =========================================================
\clearpage
\thispagestyle{empty}
\vspace*{\fill}
\begin{center}
% (dedicatória aqui depois)
\end{center}
\vspace*{\fill}
\clearpage

% =========================================================
% Resumo (pt)
% =========================================================
\selectlanguage{portuguese}
\chapter*{Resumo}
% (texto aqui depois)

\noindent\textbf{Palavras-chave:} % palavra-chave 1; palavra-chave 2; palavra-chave 3.

% =========================================================
% Abstract (en)
% =========================================================
\selectlanguage{english}
\chapter*{Abstract}
% (texto aqui depois)

\noindent\textbf{Keywords:} % keyword 1; keyword 2; keyword 3.

% =========================================================
% Resumo Alargado em Português
% =========================================================
\selectlanguage{portuguese}
\chapter*{Resumo Alargado em Português}
% (texto aqui depois)

% =========================================================
% (No PDF aparecem listas antes do corpo, com páginas em romano)
% Contents / List of Tables / List of Figures / List of Abbreviations and Acronyms
% =========================================================
\selectlanguage{portuguese}
\tableofcontents
\clearpage

\listoftables
\clearpage

\listoffigures
\clearpage

\chapter*{Lista de Abreviações e Acrónimos}
% (estrutura sugerida; preenche depois)
\begin{description}
  \item[ABBR] Full term
\end{description}
\clearpage

% =========================================================
% MAINMATTER (numeração arábica)
% =========================================================
\pagenumbering{arabic}
\setcounter{page}{27} % no PDF, Introduction começa na p. 27

% =========================================================
% Introduction (no PDF aparece como secção principal, sem "Chapter 0")
% =========================================================
\selectlanguage{portuguese}
\chapter*{Introdução}
\addcontentsline{toc}{chapter}{Introdução}
% (texto aqui depois)

% =========================================================
% CHAPTER 1 : Contexto histórico e fundamentos teóricos
% =========================================================
\chapter{Contexto Histórico e Fundamentos Teóricos}
\section{Origens das formas normais na lógica}
\section{Desenvolvimento e importância na computação}
\section{Definições e propriedades: negativa, conjuntiva, disjuntiva e prenexa}
\section{Completude funcional: bases de conectivos}
\section{Limitações teóricas}

% =========================================================
% CHAPTER 2 : Algoritmos de Transformação
% =========================================================
\chapter{Algoritmos de Transformação}
\section{Revisão dos principais métodos de conversão para formas normais}
\section{Correção, equivalência lógica e complexidade computacional}
\section{Considerações sobre implementação computacional}

% =========================================================
% CHAPTER 3 : Projeto e implementação da ferramenta
% =========================================================
\chapter{Projeto e Implementação da Ferramenta}
\section{Requisitos funcionais e arquitetura da solução}
\section{Módulos de transformação algorítmica}
\section{Interface pedagógica}
\section{Validação: testes automatizados}

% =========================================================
% CHAPTER 4 : Conclusão
% =========================================================
\chapter{Conclusão}
\section{Síntese dos resultados obtidos}
\section{Contribuições para o ensino de lógica}
\section{Limitações do trabalho realizado}
\section{Perspetivas de evolução e trabalhos futuros}

% =========================================================
% Conclusion (no PDF aparece como secção final, não numerada)
% =========================================================
\chapter*{Conclusion}
\addcontentsline{toc}{chapter}{Conclusion}
% (texto aqui depois)

% =========================================================
% References (no PDF: "References")
% =========================================================
\chapter*{References}
\addcontentsline{toc}{chapter}{References}
% (bibliografia aqui depois — BibTeX/Biber ou manual)

% =========================================================
% APPENDICES I–X (como no PDF)
% =========================================================
\appendix

\chapter{Scripts de Transformação}
\section{Scripts de Transformação}

\end{document}
