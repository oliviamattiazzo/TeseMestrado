\documentclass[12pt,a4paper,oneside]{report}

% -------------------------
% Pacotes essenciais
% -------------------------
\usepackage[utf8]{inputenc}
\usepackage[T1]{fontenc}
\usepackage[english,portuguese]{babel}
\usepackage{lmodern}
\usepackage{geometry}
\usepackage{setspace}
\usepackage{graphicx}
\usepackage{hyperref}

% (Opcional, mas comum em dissertações)
\usepackage{amsmath,amssymb,amsthm}

% -------------------------
% Margens (ajusta se o template UAb pedir outras)
% -------------------------
\geometry{
  left=3cm,
  right=2cm,
  top=3cm,
  bottom=2.5cm
}

\onehalfspacing
\hypersetup{
  colorlinks=true,
  linkcolor=black,
  citecolor=black,
  urlcolor=black
}

% =========================================================
% Metadados (preenche depois)
% =========================================================
\newcommand{\UAbUniversity}{UNIVERSIDADE ABERTA}
\newcommand{\DocTitle}{Formas Normais: da Lógica Matemática à Computação}
\newcommand{\AuthorName}{Olivia Pachele Mattiazzo}
\newcommand{\DegreeName}{Mestrado em Estatística, Matemática e Computação}
\newcommand{\OrientadorUm}{Prof.\textsuperscript{ª} Dr.\textsuperscript{ª} Gilda Ferreira}
\newcommand{\OrientadorDois}{Prof.\textsuperscript{ª} Dr.\textsuperscript{ª} Yves Robert}
\newcommand{\DocMonthYear}{Novembro 2026}

% =========================================================
\begin{document}

% =========================================================
% CAPA 1 (com orientação / coorientação)
% =========================================================
\begin{titlepage}
  \selectlanguage{english}
  \thispagestyle{empty}
  \begin{center}
    {\Large \UAbUniversity \par}
    \vspace{2.5cm}
    {\Large \textbf{\DocTitle}\par}
    \vspace{0.5cm}
    {\Large \AuthorName\par}
    \vspace{0.8cm}
    {\Large \DegreeName\par}
    \vspace{1.2cm}
    {\large Orientadores: \OrientadorUm\par}
    {\large e \OrientadorDois\par}
    \vfill
    {\Large \DocMonthYear\par}
  \end{center}
\end{titlepage}

% =========================================================
% CAPA 2 (sem orientação/coorientação)
% =========================================================
\begin{titlepage}
  \selectlanguage{portuguese}
  \thispagestyle{empty}
  \begin{center}
    {\Large \UAbUniversity \par}
    \vspace{2.5cm}
    {\Large \textbf{\DocTitle}\par}
    \vspace{0.5cm}
    {\Large \DocSubtitle\par}
    \vspace{2.5cm}
    {\Large \AuthorName\par}
    \vspace{0.8cm}
    {\Large \DegreeName\par}
    \vfill
    {\Large \DocMonthYear\par}
  \end{center}
\end{titlepage}

% =========================================================
% FRONTMATTER (numeração romana, como no PDF)
% =========================================================
\pagenumbering{roman}
\setcounter{page}{2} % no PDF, a página ii é a licença CC

% =========================================================
% Creative Commons License (página ii no PDF)
% =========================================================
\selectlanguage{english}
\chapter*{Creative Commons License}
% (texto da licença aqui depois)

% =========================================================
% Acknowledgments (página iii)
% =========================================================
\chapter*{Acknowledgments}
% (texto aqui depois)

% =========================================================
% Página em branco / separador (se precisares)
% \clearpage\null\thispagestyle{empty}\clearpage

% =========================================================
% Dedicatória (no PDF: página com dedicatória curta)
% =========================================================
\clearpage
\thispagestyle{empty}
\vspace*{\fill}
\begin{center}
% (dedicatória aqui depois)
\end{center}
\vspace*{\fill}
\clearpage

% =========================================================
% Resumo (pt)
% =========================================================
\selectlanguage{portuguese}
\chapter*{Resumo}
% (texto aqui depois)

\noindent\textbf{Palavras-chave:} % palavra-chave 1; palavra-chave 2; palavra-chave 3.

% =========================================================
% Abstract (en)
% =========================================================
\selectlanguage{english}
\chapter*{Abstract}
% (texto aqui depois)

\noindent\textbf{Keywords:} % keyword 1; keyword 2; keyword 3.

% =========================================================
% Resumo Alargado em Português
% =========================================================
\selectlanguage{portuguese}
\chapter*{Resumo Alargado em Português}
% (texto aqui depois)

% =========================================================
% (No PDF aparecem listas antes do corpo, com páginas em romano)
% Contents / List of Tables / List of Figures / List of Abbreviations and Acronyms
% =========================================================
\selectlanguage{english}
\tableofcontents
\clearpage

\listoftables
\clearpage

\listoffigures
\clearpage

\chapter*{List of Abbreviations and Acronyms}
% (estrutura sugerida; preenche depois)
\begin{description}
  \item[ABBR] Full term
\end{description}
\clearpage

% =========================================================
% MAINMATTER (numeração arábica)
% =========================================================
\pagenumbering{arabic}
\setcounter{page}{27} % no PDF, Introduction começa na p. 27

% =========================================================
% Introduction (no PDF aparece como secção principal, sem "Chapter 0")
% =========================================================
\selectlanguage{english}
\chapter*{Introduction}
\addcontentsline{toc}{chapter}{Introduction}
% (texto aqui depois)

% =========================================================
% CHAPTER 1 : NUMERAL SYSTEMS
% =========================================================
\chapter{Numeral Systems}
\section{Introduction to Numeral Systems}
\section{Decimal System}
\section{Binary System}
\section{Octal and Hexadecimal Systems}
\section{Numeral Systems and Boolean Algebra}
\section{Conclusion}

% =========================================================
% CHAPTER 2 : BOOLEAN ALGEBRA
% =========================================================
\chapter{Boolean Algebra}
\section{Introduction to Boolean Algebra}
\section{Definition and Fundamental Concepts}
\section{Boolean Algebraic Identities and Properties}
\section{De Morgan’s Laws}
\section{Algebraic Simplification of Boolean Expressions}
\section{Relationship with Set Theory}
\section{Conclusion}

% =========================================================
% CHAPTER 3 : LOGIC GATES
% =========================================================
\chapter{Logic Gates}
\section{Introduction to Logic Gates in Digital Circuits}
\section{Logic Gates}
\section{Types of Logic Gates}
\section{Conclusion}

% =========================================================
% CHAPTER 4 : MODELLING OF DIGITAL CIRCUITS
% =========================================================
\chapter{Modelling of Digital Circuits}
\section{Introduction to the Modelling of Digital Circuits}
\section{Minterms and Maxterms}
\section{Sum of Products}
\section{Product of Sums}
\section{Karnaugh Maps}
\section{Karnaugh’s Minimization}
\section{Conclusion}

% =========================================================
% CHAPTER 5 : COMBINATIONAL CIRCUITS
% =========================================================
\chapter{Combinational Circuits}
\section{Introduction to Combinational Circuits}
\section{Adders and Subtractors}
\section{Decoders and Encoders}
\section{Conclusion}

% =========================================================
% CHAPTER 6 : SEQUENTIAL CIRCUITS
% =========================================================
\chapter{Sequential Circuits}
\section{Introduction to Sequential Circuits}
\section{Latches}
\section{Flip-flops}
\section{Binary Counter}
\section{Conclusion}

% =========================================================
% CHAPTER 7 : NEURAL NETWORKS AND DEEP LEARNING
% =========================================================
\chapter{Neural Networks and Deep Learning}
\section{Introduction to Deep Learning}
\section{The Perceptron}
\section{General Concepts}
\section{Feed Forward Networks (FFNs)}
\section{Gradient-Driven Optimization}
\section{Backpropagation Algorithm}
\section{Conclusion}

% =========================================================
% CHAPTER 8 : LOGIC GATE NEURAL NETWORKS
% =========================================================
\chapter{Logic Gate Neural Networks}
\section{Introduction}
\section{Logic Gate Networks (LGNs)}
\section{Differentiable Logic Gate Networks}
\section{Methodology}
\section{Training an LGN}
\section{Training an FFN}
\section{Resources}
\section{Discussion}
\section{Conclusion}

% =========================================================
% Conclusion (no PDF aparece como secção final, não numerada)
% =========================================================
\chapter*{Conclusion}
\addcontentsline{toc}{chapter}{Conclusion}
% (texto aqui depois)

% =========================================================
% References (no PDF: "References")
% =========================================================
\chapter*{References}
\addcontentsline{toc}{chapter}{References}
% (bibliografia aqui depois — BibTeX/Biber ou manual)

% =========================================================
% APPENDICES I–X (como no PDF)
% =========================================================
\appendix

\chapter{Octal and Hexadecimal Systems}
\section{Octal and Hexadecimal Systems}

\chapter{Boolean Algebra and Set Theory}
\section{Boolean Algebra and Set Theory}

\chapter{Karnaugh’s Minimization}
\section{Karnaugh’s Minimization for a Four-Variable Function}

\chapter{Equations of the FA}
\section{Equations of the FA}

\chapter{Subtraction with the 2’s Complement Method}
\section{Subtraction with the 2’s Complement Method}

\chapter{Decoders and Encoders}
\section{Decoders and Encoders}

\chapter{Gradient Descent}
\section{Gradient Descent}

\chapter{Backpropagation Algorithm}
\section{Backpropagation Algorithm}

\chapter{Probabilistic T-norm and T-conorm Operations}
\section{Probabilistic T-norm and T-conorm Operations}

\chapter{Script in Python}
\section{Script in Python}

\end{document}
